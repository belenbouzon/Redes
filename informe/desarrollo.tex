\section{Detalles de implementación}

Con el fin de llevar a cabo el desarrollo del presente trabajo práctico desarrollamos una serie de algoritmos que hacen uso de las bibliotecas Scapy, un manipulador de paquetes interactivo escrito en Python. Uno de ellos, \textit{escuchar2.py} se encarga de escuchar la red en modo promiscuo y generar un archivo *.pcap con la información de los paquetes transmitidos.  Luego, \textit{cuantificacionTiposDePaquetes.py} y \textit{procesarArp.py} \textcolor{red}{OJO. VER SI EFECTIVAMENTE QUEDO EN ESE ARCHIVO} toman como input - entre otros parámetros - dichos archivos y otorgan información de dos tipos: el primero informa sobre la cantidad de paquetes observados de cada tipo posible, la información que cada tipo provee y la entropía de la fuente. El segundo, filtra únicamente los paquetes de tipo ARP y calcula cuántas veces cada IP fue requerida en un paquete \textit{who is}, en cuántas ocasiones cada IP envió mensajes de dicho tipo y cuál es la entropía de cada una de las fuentes. \\
Por otro lado, en el archivo \textit{utils.py} se implementó un mapeo que hace explícita la relación entre un tipo y su representación en valor hexadecimal. \\

Una serie de gráficos fue generada a partir de los datos obtenidos con el fin de presentarlos de una manera más transparente y de mejor lectura. En ellos se puede distinguir rápidamente la entropía de cada fuente, los distintos valores observados y los nodos que pueden considerarse significativos en cuanto a valor por encima de la entropía (aquellos cuya ocurrencia no es frecuente) o por debajo de la misma (aquellos que aportan poca información debido a su alta frecuencia de aparición).\\
Otro conjunto de gráficos mostrará, para cada fuente, un grafo asociado en donde se verán los vínculos establecidos entre los diversos hosts. Esta forma alternativa de presentar los resultados permitirá exponer un panorama amplio de la red observada que permita una comprensión más intuitiva del funcionamiento de la misma.\\

Todos los resultados obtenidos y los gráficos realizados servirán a los fines de generar el análisis que se extiende en la próxima sección.
