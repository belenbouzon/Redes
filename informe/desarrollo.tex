\section{Detalles de implementaci\'on}

Con el fin de llevar a cabo el desarrollo del presente trabajo pr\'actico desarrollamos una serie de algoritmos que hacen uso de las bibliotecas Scapy, un manipulador de paquetes interactivo escrito en Python. Uno de ellos, \textit{escuchar2.py} se encarga de escuchar la red en modo promiscuo y generar un archivo *.pcap con la informaci\'on de los paquetes transmitidos.  Luego, \textit{cuantificacionTiposDePaquetes.py} y \textit{procesarArp.py} toman como input - entre otros par\'ametros - dichos archivos y otorgan informaci\'on de dos tipos: el primero informa sobre la cantidad de paquetes observados de cada tipo posible, la informaci\'on que cada tipo provee y la entrop\'ia de la fuente. El segundo, filtra \'unicamente los paquetes de tipo ARP y calcula cu\'antas veces cada IP fue requerida en un paquete \textit{who is}, en cu\'antas ocasiones cada IP envi\'o mensajes de dicho tipo y cu\'al es la entrop\'ia de cada una de las fuentes. \\
Por otro lado, en el archivo \textit{utils.py} se implement\'o un mapeo que hace expl\'icita la relaci\'on entre un tipo y su representaci\'on en valor hexadecimal. \\

Una serie de gr\'aficos fue generada a partir de los datos obtenidos con el fin de presentarlos de una manera m\'as transparente y de mejor lectura. En ellos se puede distinguir r\'apidamente la entrop\'ia de cada fuente, los distintos valores observados y los nodos que pueden considerarse significativos en cuanto a valor por encima de la entrop\'ia (aquellos cuya ocurrencia no es frecuente) o por debajo de la misma (aquellos que aportan poca informaci\'on debido a su alta frecuencia de aparici\'on).\\
Otro conjunto de gr\'aficos mostrar\'a, para cada fuente, un grafo asociado en donde se ver\'an los v\'inculos establecidos entre los diversos hosts. Esta forma alternativa de presentar los resultados permitir\'a exponer un panorama amplio de la red observada que permita una comprensi\'on m\'as intuitiva del funcionamiento de la misma.\\

Todos los resultados obtenidos y los gr\'aficos realizados servir\'an a los fines de generar el an\'alisis que se extiende en la pr\'oxima secci\'on.
