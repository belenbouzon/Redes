\section{Introducci\'on}



\subsection{Objetivo}

El objetivo propuesto por la c\'atedra fue el de utilizar t\'ecnicas provistas por la teor\'ia de la informaci\'on para distinguir
diversos aspectos de una variedad de redes de manera anal\'itica. 
Para ello, nos fue solicitada la implementaci\'on de un conjunto de herramientas que nos posibilitaran capturar, manipular y analizar paquetes de informaci\'on haciendo hincapi\'e en la observaci\'on de los s\'imbolos distinguidos que pudi\'esemos notar en cada una de las fuentes propuestas y en la entrop\'ia de cada una de ellas.

\subsection{Marco te\'orico}

Dada la consigna propuesta por la c\'atedra, existen dos conceptos centrales que resulta pertinente comprender antes de continuar con la lectura del presente informe. Estos son, el de Entrop\'ia y el de protocolos ARP.

En Teor\'ia de la informaci\'on se conoce a la entrop\'ia como el promedio de informaci\'on contenida en cada mensaje enviado por una fuente. 
Como la cantidad de informaci\'on est\'a determinada en funci\'on de la probabilidad de aparici\'on de cada s\'imbolo posible (siendo menor la informaci\'on aportada cuando el s\'imbolo en cuesti\'on no es distinguido) se desprende l\'ogicamente que la entrop\'ia ser\'a inversamente proporcional a la predictibilidad  de la fuente. Es decir, que cuanto m\'as predecible sea la fuente, menor ser\'a su entrop\'ia (y viceversa).
Dicho de otra manera, cuanto menos probable sea un evento, mayor informaci\'on proporcionar\'a su aparici\'on.
 La f\'ormula para calcular la entrop\'ia de una fuente con \textit{n} eventos probables es:

 $$H(S) = \sum \limits_{i=1}^n p_i * (- \log{p_i})$$
 \newline

Por otro lado, las siglas ARP hacen referencia a \textit{Address Resolution Protocol}. Como su nombre lo indica, se trata de un protocolo de la capa \textit{data link} responsable de resolver la direcci\'on de hardware (MAC) que corresponde a una determinada direcci\'on IP, conocida por el emisor.

Para que esto sea posible, es necesario que se puedan identificar dos tipos de paquetes ARP: 

\begin{itemize}
    \item Los paquetes \textit{who-is}, que son distribuidos a todos los equipos de la red mediante broadcast con el fin de poder solicitar su MAC al nodo que posea la IP indicada en el paquete.
    \item Los paquetes \textit{is-at} transmiten mediante unicast la direcci\'on MAC solicitada - entre otras cosas - al equipo que inici\'o la comunicaci\'on solicit\'andola.
\end{itemize}

Todos los paquetes ARP contienen diversos campos que especifican:
\begin{itemize}
    \item Tipo de hardware
    \item Tipo de protocolo
    \item Longitud direcci\'on de hardware
    \item Longitud direcci\'on de protocolo
    \item C\'odigo de operaci\'on
    \item Direcci\'on hardware del emisor
    \item Direcci\'on IP del emisor
    \item Direcci\'on hardware del receptor
    \item Direcci\'on IP del receptor
\end{itemize}

En este trabajo de investigaci\'on nos enfocaremos principalmente en los \'ultimos cinco.
